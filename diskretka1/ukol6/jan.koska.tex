\documentclass{article}
\usepackage{amsmath,amssymb,amsthm}

\newtheorem{uloha}{Úloha}
\newtheorem{theorem}{Věta}
\newtheorem{lemma}{Lemma}
\renewcommand{\proofname}{Důkaz}
\newcommand{\R}{\mathbb{R}}
\newcommand{\N}{\mathbb{N}}
\newcommand{\fallingfactorial}[2]{#1^{\underline{#2}}}

\newenvironment{reseni}{\noindent\textbf{Řešení.}\hspace{0.5em}}{\hfill\qed\medskip}

\begin{document}
\begin{uloha}
Mějme $n, k \in \N$ splňující $n \geq k \geq 3$. Kolik je kružnic délky $k$ v grafu $K_n$?
\end{uloha}
\begin{reseni}
Kružnice definujeme kombinací volby $k$-tice z $n$ vrcholů a vhodným zvolením $k$ hran mezi nimi.
Možností volby $k$-tice $C \subseteq V(K_n), |C|=k$ je $n \choose k$. 
Všimneme si, že každá volba $C$ popisuje více kružnic. Každá z nich vznikne
z nějaké permutace $C$. Potřebujeme tedy spočítat počet permutací $C$ a vydělit počtem permutací, které popisují stejnou kružnici.
Počet permutací popisující jednu kružnici je $2k$, protože každé dvě permutace, které jsou navzájem cyklicky posunuté, nebo jsou navzájem v opačném pořadí, popisují stejnou kružnici.
Celkem tedy máme $\frac{1}{2k}k! = \frac{1}{2} (k-1)!$ kružnic na $k$ vrcholech a 
$$
\frac{k!}{2k}\binom{n}{k} =
\frac{(k-1)!}{2}\binom{n}{k} 
$$
kružnic délky $k$ v grafu $K_n$.
\end{reseni}
\begin{uloha}
Dokažte, že pro každý graf, jehož všechny vrcholy mají stupeň alespoň $n$, existuje cesta délky $n$.
\end{uloha}
\begin{reseni}
Postupně dokážeme existenci cesty délky $m$ pro $m=0$ až $m=n$.
Jelikož graf musí obsahovat nějaké vrcholy, cestu délky 0 máme vždy.
Předpokládejme, že pro $m \in \{0, \dots, n-1 \}$ existuje cesta $P$ délky $m$.
Nechť $v\in P, \deg_P(v)=1$ je nějaký vrchol na okraji $P$. 
Vrchol $v$ má stupeň alespoň $n$, a musí mít alespoň jednu hranu mimo $P$, 
jelikož $P$ je kratší než $n$. Touto hranou rozšíříme $P$ o jeden vrchol,
a dostaneme cestu délky $m+1$. Indukcí jsme dokázali, že pro každé $m \in \{0, \dots, n\}$
existuje cesta délky $m$ a tedy i cesta délky $n$.
\end{reseni}
\end{document}
