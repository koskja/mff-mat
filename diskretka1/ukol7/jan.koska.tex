\documentclass{article}
\usepackage{amsmath,amssymb,amsthm}

\newtheorem{uloha}{Úloha}
\newtheorem{theorem}{Věta}
\newtheorem{lemma}{Lemma}
\renewcommand{\proofname}{Důkaz}
\newcommand{\R}{\mathbb{R}}
\newcommand{\N}{\mathbb{N}}
\newcommand{\fallingfactorial}[2]{#1^{\underline{#2}}}

\newenvironment{reseni}{\noindent\textbf{Řešení.}\hspace{0.5em}}{\hfill\qed\medskip}

\begin{document}
\begin{uloha}
Nechť $T$ je strom s aspoň dvěma vrcholy takový, že pro každou jeho hranu $e$ mají obě komponenty vzniklé odebráním $e$ lichý počet vrcholů. Dokažte, že všechny vrcholy $T$ mají lichý stupeň.
\end{uloha}
\begin{reseni}
Nejprve ukážeme, že každý takový graf má sudý počet vrcholů. Nechť $e$ je libovolná hrana grafu $T$. Označme $T_1$ a $T_2$ komponenty vzniklé odebráním $e$. Jelikož $T_1$ a $T_2$ mají lichý počet vrcholů, má $T$ sudý počet vrcholů.

Nyní pro spor předpokládejme, že existuje vrchol $v$ stupně $2k$ pro nějaké $k \in \N$. 
Odebráním všech hran incidentních s $v$ získáme $2k$ komponent (mimo $v$), kde každá z nich má lichý počet vrcholů. 
Celkem máme sudý počet vrcholů v komponentách bez $v$ a celkový počet vrcholů by byl lichý, což je spor.

\end{reseni}
\begin{uloha}
Kolik koster má úplný bipartitní graf $K_{2,n}$?
\end{uloha}
\begin{reseni}
Označme vrcholy levé partity $\{a, b\}$ a pravé $\{1, 2, \dots, n\}$. Každá kostra musí obsahovat
vrcholy $a$ a $b$. 

Ukážeme, že $a, b$ jsou propojeny přes jeden prvek v pravé partitě.
Vrcholy $a$ a $b$ nemohou být propojeny přímo, jelikož leží v jedné partitě.
Označme $A$ všechny vrcholy sousedící s $a$ a $B$ všechny vrcholy sousedící s $b$.
Jelikož $A, B$ jsou v pravé partitě, nemůže mezi nimi být hrana. Jelikož kostra je souvislý graf, musí existovat
nějaký vrchol $v \in B$, který má hranu s $a$.

V každé kostře tedy existuje právě jeden vrchol $v$ v pravé partitě, který má hranu s $a$ i $b$.
Každý ze zbylých $n-1$ vrcholů v pravé partitě může být spojen s $a$ nebo $b$.
$v$ lze zvolit $n$ způsoby a pro každý zbylý vrchol máme $2$ možnosti.
Celkem tedy existuje $n\cdot2^{n-1}$ koster.

\end{reseni}
\end{document}
