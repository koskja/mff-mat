\documentclass{article}
\usepackage{amsmath,amssymb,amsthm}

\theoremstyle{definition}
\newtheorem{uloha}{Úloha}
\theoremstyle{plain}
\newtheorem{theorem}{Věta}
\newtheorem{lemma}{Lemma}
\renewcommand{\proofname}{Důkaz}
\newcommand{\R}{\mathbb{R}}
\newcommand{\N}{\mathbb{N}}
\newcommand{\Z}{\mathbb{Z}}
\newcommand{\fallingfactorial}[2]{#1^{\underline{#2}}}
\newcommand{\LO}{\text{LO\hspace{0.1em}}}

\newenvironment{reseni}{\noindent\textbf{Řešení.}\hspace{0.5em}}{\hfill\qed\medskip}

\begin{document}
\begin{uloha}
Určete počet posloupností $(v_1, v_2, v_3)$ v prostoru $\mathbb{Z}_3^3$, které jsou lineárně nezávislé.
\end{uloha}
\begin{reseni}
Bonusová úloha je srovnatelně těžká, takže ji vyřešíme a použijeme výsledek.
\begin{lemma}
Nechť $p, k, l \in \N$ a $p$ je prvočíslo. Pak v prostoru $\mathbb{Z}_p^k$ existuje $\prod_{i=1}^{l} (p^k-p^{i-1})$ lineárně nezávislých posloupností $(v_1, \hdots, v_l)$
\end{lemma}
\begin{proof}
Označme $n(l)$ počet lineárně nezávislých posloupností $(v_1, \hdots, v_l)$.
Použijeme indukci na $l$. Pro $l = 1$ je tvrzení zřejmé - v prostoru $\mathbb{Z}_p^k$ existuje $p^k$ posloupností $(v_1)$ a mezi nimi je právě $p^k-1$ lineárně nezávislých (posloupnost $(0)$ není lineárně nezávislá).
Nyní předpokládejme, že tvrzení platí pro $l-1$. 


Chtěli bychom ukázat, že každá lineárně nezávislá posloupnost $(v_1, \hdots, v_l)$ odpovídá právě jedné volbě
páru lineárně nezávislé posloupnosti $(v_1, \hdots, v_{l-1})$ a vektoru $v_l$ takového, že $v_l \notin \LO (v_1, \hdots, v_{l-1})$.

Posloupnost $(v_i)_l$ určuje pár $((v_i)_{l-1}, v_l)$ jednoznačně. O tom, že pár splňuje kýžené podmínky, se lze přesvědčit následovně: kdyby $v_l \in \LO (v_i)_{l-1}$, pak by $v_l$ byl lineární kombinací
$(v_i)_{l-1}$ a tedy $(v_i)_l$ by nebyla lineárně nezávislá. 

Nyní ukážeme, že každý pár $((v_i)_{l-1}, v_l)$ splňující dané podmínky odpovídá právě jedné lineárně nezávislé posloupnosti $(v_1, \hdots, v_l)$.
Posloupnost je určena jednoznačně ("slepením" posloupnosti $(v_i)_{l-1}$ a vektoru $v_l$). Pro spor předpokládejme, že $(v_i)_l$ je lineárně závislá. Pak $v_l \in \LO (v_i)_{l-1}$, což je spor s definicí $v_l$.

Zbývá zjistit, kolik existuje vektorů $v_l \notin \LO (v_i)_{l-1}$. Chceme určit velikost množiny $\Z_p^k \setminus \LO (v_i)_{l-1}$. 
Víme, že mezi lineárním obalem $m$ (lineárně nezávislých) vektorů a prostorem $\Z_p^m$ existuje bijekce. Pak platí 
$$ |\Z_p^k \setminus \LO (v_i)_{l-1}| = |\Z_p^k| - |\LO (v_i)_{l-1}| = |\Z_p^k| - |\Z_p^{l-1}| = p^k - p^{l-1}. $$
Jelikož počet možností volby $v_l$ nezávisí na volbě $(v_i)_{l-1}$, je počet lineárně nezávislých posloupností $(v_1, \hdots, v_l)$ roven
$$ n(l) = n(l-1) \cdot (p^k - p^{l-1}) $$
, z čehož plyne rovnost
$$ n(l) = \prod_{i=1}^{l} (p^k - p^{i-1}) $$
\end{proof}
V kontextu lemmatu 1 máme $p = 3$, $k = 3$ a $l = 3$, takže počet lineárně nezávislých posloupností $(v_1, v_2, v_3)$ je
$$ n(3) = \prod_{i=1}^{3} (3^3 - 3^{i-1}) = (3^3 - 1) \cdot (3^3 - 3) \cdot (3^3 - 3^2)= 26 \cdot 24 \cdot 18 = 11232.$$
\end{reseni}
\begin{uloha}
Předpokládejme, že ve vektorovém prostoru $V$ nad tělesem $\R$ je $(u, v, w, z)$ lineárně nezávislá
posloupnost. Rozhodněte, zda je posloupnost $(4u + 3v + 2w + z, u + 2v + 3w + 4z, u - v + z)$ také
lineárně nezávislá ve $V$.
\end{uloha}
\begin{reseni}
Posloupnost $(4u + 3v + 2w + z, u + 2v + 3w + 4z, u - v + z)$ je lineárně nezávislá právě když jediné řešení rovnice
$\lambda_1(4u + 3v + 2w + z) + \lambda_2(u + 2v + 3w + 4z) + \lambda_3(u - v + z) = 0$ je $\lambda_1 = \lambda_2 = \lambda_3 = 0$.
Úpravou dostaneme rovnici $(4\lambda_1 + \lambda_2 + \lambda_3)u + (3\lambda_1 + 2\lambda_2 - \lambda_3)v + (2\lambda_1 + 3\lambda_2 + \lambda_3)w + (\lambda_1 + 4\lambda_2 + \lambda_3)z = 0$.
Jelikož $(u, v, w, z)$ je lineárně nezávislá posloupnost, jejich koeficienty musí být nulové. Dostáváme soustavu
\begin{align*}
4\lambda_1 + \lambda_2 + \lambda_3 &= 0 \\
3\lambda_1 + 2\lambda_2 - \lambda_3 &= 0 \\
2\lambda_1 + 3\lambda_2 + \lambda_3 &= 0 \\
\lambda_1 + 4\lambda_2 + \lambda_3 &= 0
\end{align*}
s rozšířenou maticí
\begin{align*}
\begin{pmatrix}
4 & 1 & 1 & \vline & 0 \\
3 & 2 & -1 & \vline & 0 \\
2 & 3 & 1 & \vline & 0 \\
1 & 4 & 1 & \vline & 0
\end{pmatrix}
\end{align*}
ze které po Gaussově eliminaci dostaneme
\begin{align*}
\begin{pmatrix}
1 & 4 & 1 & \vline & 0 \\
0 & 1 & \frac{2}{5} & \vline & 0 \\
0 & 0 & 1 & \vline & 0 \\
0 & 0 & 0 & \vline & 0
\end{pmatrix}.
\end{align*}
Zpětnou substitucí zjistíme, že $\lambda_1 = \lambda_2 = \lambda_3 = 0$ a že posloupnost $(4u + 3v + 2w + z, u + 2v + 3w + 4z, u - v + z)$ je lineárně nezávislá.
\end{reseni}

\end{document}
