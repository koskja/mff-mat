\documentclass{article}
\usepackage{graphicx} % Required for inserting images
\usepackage{amsfonts}
\usepackage{amsmath}

\begin{document}

Ak je postupnosť vektorov $(4u + 3v + 2w + z, u + 2v + 3w + 4z, u - v + z)$ nezávislá potom má rovnica 
\begin{equation*}
    x_1(4u + 3v + 2w + z)+x_2( u + 2v + 3w + 4z)+x_3( u - v + z) =o
\end{equation*}
iba trviálne riešenie $x_1=x_2=x_3=0$ keďže postupnosť $(u,v,w,z)$ je lineárne nezávislá a teda žiaden vektor sa nedá vyjadriť ako lineárna kombinácia zvyšných musí platiť:
\begin{align}
    4x_1u +x_2u+x_3u=o\nonumber\\
    3x_1v +2x_2v-x_3v=o\nonumber\\
    2x_1w +3x_2w=o\nonumber\\
    x_1z+4x_2z+x_3z =o\nonumber
\end{align}
vektory $u,v,w,z$ môžeme vyňať zároveň vieže že žiaden z týchto  vektorov nie je nulový pretože sú lineárne nezávislé. Dostaneme sústavu rovníc pre parametre $x_1,x_2,x_3$ ktorú si zapíšeme ako maticu a prevedieme do odstupňovaného tvaru.
\begin{align}\nonumber
    \begin{pmatrix}
        4 & 1 & 1\\
        3 & 2 & -1\\
        2& 3 & 0\\
        1& 4 & 1
    \end{pmatrix}
    \sim 
        \begin{pmatrix}
        1& 4 & 1\\
        0 & -15 & -3\\
        0 & -10 & -4\\
        0& -5& -2 
    \end{pmatrix}
    \sim
    \begin{pmatrix}
        1& 4 & 1\\
        0 & 5 & 2\\
        0 & 0 & 0\\
        0& 0& 3 
    \end{pmatrix}
    \sim
    \begin{pmatrix}
        1& 4 & 1\\
        0 & 5 & 2\\
      0& 0& 3 \\
        0 & 0 & 0
    \end{pmatrix}
\end{align}
Vidíme, že všetky stĺpce sú bázové a teda jediné riešenie sústavy bude $x_1=x_2=x_3=0$ teda postupnosť je lineárne nezávislá.
\end{document}
